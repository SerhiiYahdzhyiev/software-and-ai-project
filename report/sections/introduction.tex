\section{Introduction}

In modern software development, code reuse is a widely accepted practice that
allows developers to save time, reduce redundancy, and leverage existing solutions.
However, integrating third-party code snippets or libraries into projects is not
without risks. Issues such as hidden bugs, security vulnerabilities, and
performance inefficiencies can significantly impact the stability and
maintainability of software. Identifying these issues often requires extensive
manual review and analysis, which can be time-consuming and resource-intensive.

With the advent of Large Language Models (LLMs) and their rapid advancements in
understanding and generating human-like text, it becomes possible to automate
aspects of code analysis and summarization. This project proposes a browser
extension powered by artificial intelligence to assist developers in assessing
and understanding code snippets they encounter while browsing the web. By
leveraging the Web Extensions API \autocite{CFD} and integrating LLM-based
processing, the extension provides real-time insights into code quality,
potential errors, and optimization suggestions.

A key feature of this solution is its modularity. The extension itself handles
the UI and workflow, while the AI processing is delegated to a locally hosted
Express.js server. This server acts as an intermediary, encapsulating the core
logic of code analysis while allowing users to choose and configure their
preferred LLM adapters. This approach ensures flexibility, enabling developers
to tailor the tool to their specific needs and seamlessly integrate different
AI models without modifying the core extension logic.

By embedding AI-powered code analysis directly into the web browsing experience,
this project aims to enhance the efficiency of software development and research
processes. Developers can receive immediate feedback on code snippets they
encounter, reducing the need for exhaustive manual reviews and enabling a more
informed approach to code reuse. The following sections detail the
implementation, architecture, and evaluation of the proposed system, along with
a discussion on its advantages and potential limitations.
